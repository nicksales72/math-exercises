\documentclass[11pt, a4paper]{article}

\usepackage[utf8]{inputenc}
\usepackage[T1]{fontenc}    
\usepackage{amsmath, amssymb} 
\usepackage{parskip}
\usepackage{graphicx}       
\usepackage{hyperref}       
\usepackage{geometry}       
\geometry{margin=1in}       

\title{Geometry Exercises}
\author{Nicholas Sales}
\date{} 

\begin{document}

\maketitle 
\tableofcontents

\newpage

\section{Chapter 1}

\subsection{Exercise I-1}

\textbf{Show that the function $f : \mathbb{E}^n \rightarrow \mathbb{E}^n$ defined by $f(\vec{v}) = 2 \vec{v}$ is bijective but not an isometry.}

We first show $f$ is bijective. To show injectivity, let $\vec{x}, \vec{y} \in \mathbb{E}^n$ such that $f(\vec{x}) = f(\vec{y})$. Then:
\begin{equation*}
  2 \vec{x} = 2 \vec{y} \implies \vec{x} = \vec{y}
\end{equation*}
Since $\vec{x} = \vec{y}$, $f$ is injective. We now show $f$ is surjective, let $\vec{m} \in \mathbb{E}^n$. We aim to find some $\vec{v} \in \mathbb{E}^n$ such that $f(\vec{v}) = \vec{m}$. So:
\begin{equation*}
  2 \vec{v} = \vec{m} \implies \vec{v} = \frac{1}{2} \vec{m}
\end{equation*}
Indeed, checking this:
\begin{equation*}
  f \left ( \frac{1}{2} \vec{m} \right ) = 2 \left ( \frac{1}{2} \vec{m} \right ) = \vec{m}
\end{equation*}
This shows $f$ is surjective. Thus, since $f$ is both injective and surjective, $f$ is bijective.

To show $f$ is not an isometry, we proceed by counterexample. Let $(1, 5), (2, 3) \in \mathbb{E}^n$. For $f$ to be an isometry we must have:
\begin{align*}
  || (1, 5) - (2, 3) || &= || f(1, 5) - f(2, 3) || \\
  || (-1, 2) ||         &= || (-2, 4) || \\
  \sqrt{(-1)^2 + (2)^2} &= \sqrt{(-2)^2 + (4)^2} \\
  \sqrt{5}              &= \sqrt {20} \\
  \sqrt{5}              &= 2 \sqrt {5}
\end{align*}
which is clearly false. 

Therefore, $f$ is bijective but not an isometry. $\Box$

\section{Chapter 6}

\end{document}


