\documentclass[11pt, a4paper]{article}

\usepackage[utf8]{inputenc}
\usepackage[T1]{fontenc}    
\usepackage{amsmath, amssymb} 
\usepackage{parskip}
\usepackage{graphicx}       
\usepackage{hyperref}       
\usepackage{geometry}       
\geometry{margin=1in}       

\title{Geometry Exercises}
\author{Nicholas Sales}
\date{} 

\begin{document}

\maketitle 
\tableofcontents

\newpage

\section{Chapter 1}

\subsection{I-1}

\textbf{Show that the function $f : \mathbb{E}^n \rightarrow \mathbb{E}^n$ defined by $f(\vec{v}) = 2 \vec{v}$ is bijective but not an isometry.}

We first show $f$ is bijective. It is easy to see that $f$ is invertible by defining $f^{-1}(\vec{v}) = \frac{1}{2} \vec{v}$ since:
\[
  f^{-1}(f(\vec{v})) = f^{-1}(2 \vec{v}) = \frac{1}{2} (2 \vec{v}) = \vec{v}.
\]
Thus, $f$ is bijective.

To show $f$ is not an isometry, we proceed by counterexample. Let $(1, 5), (2, 3) \in \mathbb{E}^n$. For $f$ to be an isometry we must have:
\begin{align*}
  || (1, 5) - (2, 3) || &= || f(1, 5) - f(2, 3) || \\
  || (-1, 2) ||         &= || (-2, 4) || \\
  \sqrt{(-1)^2 + (2)^2} &= \sqrt{(-2)^2 + (4)^2} \\
  \sqrt{5}              &= \sqrt {20} \\
  \sqrt{5}              &= 2 \sqrt {5},
\end{align*}
which is clearly false. 

Therefore, $f$ is bijective but not an isometry. $\Box$

\section{Chapter 6}

\end{document}


